\newglossaryentry{fork}
{
    name={Fork},
    text={fork},
    description={
        dosłownie (z ang. rozwidlenie), mówi o sytuacji,
        gdy twórcy projektu informatycznego
        decydują się na jego podzielenie i dalszy rozwój
        w dwóch różnych kierunkach}
}

\newglossaryentry{siteswap}
{
    name={Siteswap},
    text={siteswap},
    description={
        wymyślona w 1985 roku notacją pozwalającą
        zapisywać większość żonglerskich trików za pomocą cyfr}
}

\newglossaryentry{noktowizor}
{
    name={Noktowizor},
    text={noktowizor},
    description={
        to urządzenie umożliwiające widzenie w ciemności}
}

 \newglossaryentry{open source}
 {
     name={Open source},
     text={open source},
     description={
        termin określający oprogramowanie, które może być
        uruchamiane, kopiowane, rozpowszechniane, analizowane
        oraz zmieniane i poprawiane przez użytkowników}
 }

\newglossaryentry{kanal alpha}
{
    name={Kanał alpha},
    text={kanał alpha},
    ale={kanale alpha},
    description={
    dodatkowy kanał, definiujący przezroczystość
    wyświetlanych informacji graficznych}
}

\newglossaryentry{standalone}
{
    name={Standalone},
    text={standalone},
    description={
      program komputerowy, który nie ładuje żadnego zewnętrznego
      modułu, biblioteki lub funkcji i jest zaprojektowany
      żeby uruchomić się w konkretnym środowisku}
}

\newglossaryentry{kolejka}
  {
      name={Kolejka},
      text={kolejka},
      description={
        liniowa struktura danych, w której nowe dane
        dopisywane są na końcu, a pobierane z początku,
           do dalszego przetwarzania}
  }


 \newglossaryentry{slownik}
 {
     name={Tablica asocjacyjna},
     text={tablica asocjacyjna},
     ku={słowniku},
     description={
         (słownik, mapa,tablica skojarzeniowa)
        abstrakcyjnego typu danych, który przechowuje
        pary (unikatowy klucz, wartość) i umożliwia
        dostęp do wartości poprzez podanie klucza}
 }

\newglossaryentry{watek}
{
    name={Wątek},
    text={wątek},
    description={
        część programu wykonywana współbieżnie
        w obrębie jednego procesu}
}

\newglossaryentry{allokacja}
{
    name={Allokacja},
    text={allokacja},
    wane={allokowane},
    description={
        przypisywanie zasobów do możliwości ich użycia}
}


\newglossaryentry{widget}
{
    name={Widget},
    text={widget},
    description={
        inaczej kontrolka lub element kontrolny.
    W produktach firmy Microsoft (Visual Studio, .NET, Office itp.)
     używana jest nazwa formatka.
      W pewnych kontekstach synonimem widżetu
      jest okno}
}

\newglossaryentry{kalibracja}
{
    name={Kalibracja},
    text={kalibracja},
    description={
        proces dopasowywania parametrów
     urządzenia do warunków pracy}
}


\newglossaryentry{stream}
{
    name={Stream},
    text={stream},
    description={
        seria danych, której kolejne elementy
        oddziela zmienny interwał czasu}
}


\newglossaryentry{prymitywy}
{
    name={Prymitywy},
    text={prymitywy},
    description={
         rodzaj figur geometrycznych w grafice komputerowej, z
        których buduje się inne, bardziej skomplikowane.
       Z punktu widzenia geometrycznej definicji figury,
       każdą z nich można zbudować z punktów. W grafice
        komputerowej najczęściej jednak jako prymitywów
        używa się trójkątów (szczególnie w grafice trójwymiarowej), albo odcinków.
       Inne figury, często stosowane jako prymitywy:
       różnego rodzaju krzywe, okręgi, koła, sfery, kwadraty itp}
}


\newglossaryentry{real time computing}
{
    name={Real Time Computing},
    wczym={aplikacjach przetwarzających dane w czasie rzeczywistym},
    description={
        opisuje aplikacje lub urządzenia, których wynik musi być
        zagwarantowany w określonym czasie}
}



\newglossaryentry{sterta}
{
    name={Sterta},
    text={Sterta},
    cie={stercie},
    description={
        obszar pamięci, udostępniony na wyłączność uruchomionemu programowi
         (procesowi). Przechowuje się tam zmienne dynamiczne}
}

\newglossaryentry{wycieki pamieci}
{
    name={Wycieki pamięci},
    text={wycieki pamięci},
    ow={wycieków pamięci},
    cie={stercie},
    description={
        szczególny rodzaj niezamierzonego użycia pamięci
        przez program komputerowy, gdy nie zwalnia on zaalokowanej
        wcześniej pamięci, która nie jest już mu potrzebna,
        a może nawet rezerwuje nową}
}


\newglossaryentry{system czasu rzeczywistego}
{
    name={Real Time Operating System},
    description={
        (system operacyjny czasu rzeczywistego)
        komputerowy system operacyjny, który został opracowany
         tak, by spełnić wymagania narzucone na czas wykonywania
         zadanych operacji. Systemy takie stosuje się jako elementy
          komputerowych systemów sterowania pracujących w reżimie
          czasu rzeczywistego - system czasu rzeczywistego}
}





